\documentclass{article}
\usepackage[spanish]{babel}
\usepackage{amsmath}

\title{Ejemplo básico con Tablas de LaTeX}
\author{Erick Diaz}
\date{\today}

\begin{document}
\maketitle

\section{Ecuaciones en LaTeX}

\subsection{Ecuaciones en Linea}
La famosa ecuación $E = mc^2$ relaciona masa y energía.

\subsection{Ecuaciones centradas (sin numerar)}
\[
a^2 + b^2 = c^2 \\
\]
\[
\sum_{i=1}^{n+1} i
\]

\subsection{Ecuaciones numeradas}
\begin{equation*}
F = ma
\label{eq:newton}
\end{equation*}
La ecuación \ref{eq:newton} representa la segunda ley de Newton.

\subsection{Fracciones, raíces y potencias}
\[
x = \frac{-b \pm \sqrt{b^2 - 4ac}}{2a}
\]

\subsection{Sumatorias e integrales}
\[
\sum_{i=1}^{n} i = \frac{n(n+1)}{2}
\]
\[
\int_{0}^{\infty} e^{-x}\,dx = 1
\]

\subsection{Sistemas de ecuaciones}
\[
\begin{cases}
x + y = 10 \\
2x - y = 5
\end{cases}
\]

\subsection{Ecuaciones alineadas}
\begin{align}
f(x) &= x^4 + 2x \\
     &= (x + 1)^2
\end{align}

\subsection{Matrices}
\[
A =
\begin{pmatrix}
1 & 2 \\
3 & 4
\end{pmatrix}
\]

\end{document}